\documentclass[a4paper, 10pt, ngerman]{article}
\usepackage{tikz-network}
\usepackage[left=3cm, right = 3cm, top=3cm, bottom=3cm]{geometry}
\usepackage{babel}
\usepackage[T1]{fontenc}
\usepackage{inputenc}
\usepackage{algpseudocode}
\usepackage{amsmath}
\usepackage{amsthm}
\usepackage[nottoc,notlot,notlof]{tocbibind}
\usepackage{graphicx}
\usepackage{svg}

\newcommand{\Aufgabe}{Aufgabe 3: Pancake Sort}
\newcommand{\TeilnahmeId}{67571}
\newcommand{\Name}{Finn Rudolph}

\usepackage{scrlayer-scrpage, lastpage}
\setkomafont{pageheadfoot}{\textrm}
\rohead{Teilnahme-ID: \TeilnahmeId}
\lohead{\Aufgabe}
\cfoot*{\thepage{}}

\title{\LARGE \textbf{Aufgabe 3: Pancake Sort}}
\author{\large Finn Rudolph \\ \\ \large Teilnahme-ID: 67571}
\date{\large 28. Dezember 2022}

\begin{document}

\begin{titlepage}
    \maketitle
    \tableofcontents
    \thispagestyle{empty}
\end{titlepage}

\newtheorem{theorem}{Satz}
\newtheorem{lemma}{Lemma}
\theoremstyle{definition}
\newtheorem{definition}{Definition}

\section{Lösungsidee}

Pancake Sort ist ein bekannter Sortieralgorithmus, in seiner ursprünglichen Form wird der oberste Pfannkuchen allerdings nicht aufgegessen, sondern allein ein Präfix der zu sortierenden Folge umgekehrt. Pancake Sort wird daher auch als \emph{Sorting by Prefix Reversals} bezeichnet. Der Algorithmus wird zwar nicht direkt zum Sortieren verwendet, der mit ihm verbundene Pancake Graph dient jedoch beispielsweise als Grundlage für Netzwerke für parallele Computersysteme.

Ein Stapel von $n$ Pfannkuchen wird als Folge
$$
    p = p_0, p_1, p_2, \dots, p_{n - 1}
$$
definiert, wobei $1 \le p_i \le n$ und $p_i \ne p_j$, für $0 \le i, j \le n-1$ und $i \ne j$. $p_0$ ist der oberste Pfannkuchen, $p_{n - 1}$ der unterste. Folgen und Ähnliches werden grundsätzlich mit 0 beginnend indexiert. Da jede Zahl von 1 bis $n$ in dieser Folge genau einmal vorkommt, wird $p$ auch als Permutation der Länge $n$ bezeichnet. Um eine Wende-und-Ess-Operation kompakt auszudrücken, wird der Wende-und-Ess-Operator $\gamma_i$ eingeführt.

\begin{definition}
    Sei $p$ eine Permutation der ersten $n$ natürlichen Zahlen. Der Wende-und-Ess-Operator $\gamma_i$ ist für $p$ wie folgt definiert.
    $$
        \gamma_i p = p_{i-1}', p_{i-2}', \dots, p_1', p_0', p_{i+1}', p_{i+2}', \dots, p_{n-1}'
    $$
    $$
        p_j' = \begin{cases}
            p_j     & \text{wenn } p_j < p_i \\
            p_j - 1 & \text{wenn } p_j > p_i
        \end{cases}
        \quad 0 \le j \le n - 1, j \ne i
    $$
\end{definition}
$\gamma_i p$ bezeichnet also die Permutation der ersten $n - 1$ natürlichen Zahlen, die man erhält, wenn die ersten $i+1$ Elemente von $p$ umgekehrt werden und anschließend das erste entfernt wird. Um tatsächlich eine Permutation der Länge $n - 1$ zu erhalten, werden durch $\gamma_i$ außerdem alle Elemente von $p$, die größer als $p_i$ sind, um 1 verkleinert. Das erhält die relative Ordung der Elemente, sodass die optimale Folge an Wende-und-Ess-Operationen unverändert bleibt. Zum Beispiel, wenn $p = 4, 1, 2, 3$, dann ist
$$
    \gamma_2 p = \gamma_2 (4, 1, 2, 3) = 1, 3, 2
$$
Um später das in Teilaufgabe a$)$ vorliegende Problem klar benennen zu können, wird ihm der Name ``$\gamma_i$-Pancake Sort'' gegeben.

\begin{definition}[$\gamma_i$-Pancake Sort]
    Gegeben sei eine Permutation $p$ der ersten $n$ natürlichen Zahlen. Wie lautet eine kürzestmögliche Folge an $\gamma_i$-Operationen, mit denen $p$ in eine identische Permutation beliebiger Länge umgewandelt wird?
\end{definition}

\subsection{Der Pancake-Graph}

Im Kontext des ursprünglichen Pancake-Sort ist der sogenannte Pancake-Graph für Permutationen der Länge $n$ wie folgt definiert: Für jede Permutation der Länge $n$ gibt es genau einen Knoten, und zwischen den Knoten zweier Permutationen verläuft genau dann eine ungerichtete Kante, wenn sie durch das Umkehren eines Präfixes ineinander umgewandelt werden können. Analog dazu kann man einen Pancake-Graphen $G_n$ für das vorliegende Problem definieren: Jeder Permutation der Länge $n$ oder kleiner wird ein Knoten zugeordnet, und zwischen zwei Permutationen $p$ und $q$ wird eine von $p$ nach $q$ gerichtete Kante eingefügt, wenn $\gamma_i p = q$, für irgendein $0 \le i \le |p|-1$. $G_n$ ist also ein gerichteter, azyklischer Graph (DAG) mit $n$ ``Ebenen'', für jede Permutationslänge eine. Kanten verlaufen nur zur direkt folgenden Ebene, da $\gamma_i$ die Länge um genau 1 reduziert. In Abbildung 1 ist $G_4$ dargestellt, aufgrund seiner Größe wurde er auf 2 Seiten aufgeteilt und Knoten der kleinern Permutationen dupliziert. Die Knoten sind jeweils in lexikographischer Ordung von oben nach unten angeordnet. In diesem Abschnitt soll eine interessante Symmetrieeigenschaft von $G_n$ behandelt werden, die eine Optimierung der Lösungen für beide Teilaufgaben ermöglicht.

Betrachtet man die beiden Teile von $G_4$ gleichzeitig, fällt auf, dass der zweite Teil genau wie der erste Teil aussieht, nur vertikal gespiegelt. Auch die kleineren Pancake-Graphen $G_3, G_2$ und $G_1$, die Teilgraphen von $G_4$ sind, scheinen symmetrisch um ihre Mitte zu sein. Genauer sind die Kanten symmetrisch um die Mitte der lexikographisch sortierten Liste der Permutationen von Größe 4.

\newpage
\begin{tikzpicture}[node distance = {19mm}, main/.style = {draw, circle}]
    \node[main](1234) at (0, 0) {1234};
    \node[main](1243) [below of = 1234] {1243};
    \node[main](1324) [below of = 1243] {1324};
    \node[main](1342) [below of = 1324] {1342};
    \node[main](1423) [below of = 1342] {1423};
    \node[main](1432) [below of = 1423] {1432};

    \node[main](2134) [below of = 1432] {2134};
    \node[main](2143) [below of = 2134] {2143};
    \node[main](2314) [below of = 2143] {2314};
    \node[main](2341) [below of = 2314] {2341};
    \node[main](2413) [below of = 2341] {2413};
    \node[main](2431) [below of = 2413] {2431};

    \node[main](123) at (7, -5.8) {123};
    \node[main](132) [below of = 123] {132};
    \node[main](213) [below of = 132] {213};
    \node[main](231) [below of = 213] {231};
    \node[main](312) [below of = 231] {312};
    \node[main](321) [below of = 312] {321};

    \node[main](12) at (11, -9.6) {12};
    \node[main](21) [below of = 12] {21};

    \node[main](1) at (13.5, -10.5) {1};

    \draw[->](1234) -- (123);
    \draw[->](1234) -- (213);
    \draw[->](1234) -- (321);

    \draw[->](1243) -- (132);
    \draw[->](1243) -- (213);
    \draw[->](1243) -- (321);

    \draw[->](1324) -- (213);
    \draw[->](1324) -- (123);
    \draw[->](1324) -- (231);

    \draw[->](1342) -- (231);
    \draw[->](1342) -- (132);
    \draw[->](1342) -- (312);
    \draw[->](1342) -- (321);

    \draw[->](1423) -- (312);
    \draw[->](1423) -- (123);
    \draw[->](1423) -- (231);

    \draw[->](1432) -- (321);
    \draw[->](1432) -- (132);
    \draw[->](1432) -- (312);
    \draw[->](1432) -- (231);

    \draw[->](2134) -- (123);
    \draw[->](2134) -- (312);

    \draw[->](2143) -- (132);
    \draw[->](2143) -- (123);
    \draw[->](2143) -- (312);

    \draw[->](2314) -- (213);
    \draw[->](2314) -- (132);

    \draw[->](2341) -- (231);
    \draw[->](2341) -- (321);

    \draw[->](2413) -- (312);
    \draw[->](2413) -- (213);
    \draw[->](2413) -- (132);

    \draw[->](2431) -- (321);
    \draw[->](2431) -- (231);

    \draw[->](123) -- (12);
    \draw[->](123) -- (21);
    \draw[->](132) -- (21);
    \draw[->](132) -- (12);
    \draw[->](213) -- (12);
    \draw[->](231) -- (21);
    \draw[->](312) -- (12);
    \draw[->](312) -- (21);
    \draw[->](321) -- (12);
    \draw[->](321) -- (21);

    \draw[->](12) -- (1);
    \draw[->](21) -- (1);
\end{tikzpicture}
\newpage
\begin{tikzpicture}[node distance = {19mm}, main/.style = {draw, circle}]
    \node[main](3124) at (0, 0) {3124};
    \node[main](3142) [below of = 3124] {3142};
    \node[main](3214) [below of = 3142] {3214};
    \node[main](3241) [below of = 3214] {3241};
    \node[main](3412) [below of = 3241] {3412};
    \node[main](3421) [below of = 3412] {3421};

    \node[main](4123) [below of = 3421] {4123};
    \node[main](4132) [below of = 4123] {4132};
    \node[main](4213) [below of = 4132] {4213};
    \node[main](4231) [below of = 4213] {4231};
    \node[main](4312) [below of = 4231] {4312};
    \node[main](4321) [below of = 4312] {4321};

    \node[main](123) at (7, -5.8) {123};
    \node[main](132) [below of = 123] {132};
    \node[main](213) [below of = 132] {213};
    \node[main](231) [below of = 213] {231};
    \node[main](312) [below of = 231] {312};
    \node[main](321) [below of = 312] {321};

    \node[main](12) at (11, -9.6) {12};
    \node[main](21) [below of = 12] {21};

    \node[main](1) at (13.5, -10.5) {1};

    \draw[->](3124) -- (123);
    \draw[->](3124) -- (213);

    \draw[->](3142) -- (132);
    \draw[->](3142) -- (231);
    \draw[->](3142) -- (312);

    \draw[->](3214) -- (213);
    \draw[->](3214) -- (123);

    \draw[->](3241) -- (231);
    \draw[->](3241) -- (312);

    \draw[->](3412) -- (312);
    \draw[->](3412) -- (321);
    \draw[->](3412) -- (132);

    \draw[->](3421) -- (321);
    \draw[->](3421) -- (132);

    \draw[->](4123) -- (123);
    \draw[->](4123) -- (312);
    \draw[->](4123) -- (132);
    \draw[->](4123) -- (213);

    \draw[->](4132) -- (132);
    \draw[->](4132) -- (321);
    \draw[->](4132) -- (213);

    \draw[->](4213) -- (213);
    \draw[->](4213) -- (312);
    \draw[->](4213) -- (132);
    \draw[->](4213) -- (123);

    \draw[->](4231) -- (231);
    \draw[->](4231) -- (321);
    \draw[->](4231) -- (213);

    \draw[->](4312) -- (312);
    \draw[->](4312) -- (231);
    \draw[->](4312) -- (123);

    \draw[->](4321) -- (321);
    \draw[->](4321) -- (231);
    \draw[->](4321) -- (123);

    \draw[->](123) -- (12);
    \draw[->](123) -- (21);
    \draw[->](132) -- (21);
    \draw[->](132) -- (12);
    \draw[->](213) -- (12);
    \draw[->](231) -- (21);
    \draw[->](312) -- (12);
    \draw[->](312) -- (21);
    \draw[->](321) -- (12);
    \draw[->](321) -- (21);

    \draw[->](12) -- (1);
    \draw[->](21) -- (1);
\end{tikzpicture}
\begin{figure}
    \caption{Der Pancake-Graph $G_4$. Von Knoten $u$ verläuft genau dann eine gerichtete Kante nach Knoten $v$, wenn die zu $u$ zugehörige Permutation durch ein $\gamma_i$ in die zu $v$ zugehörige Permutation umgewandelt werden kann.}
\end{figure}
\newpage

Wenn man $G_4$ an einem Stück ohne Duplikation von Knoten zeichnen würde, wäre tatsächlich der gesamte Graph achsensymmetrisch um seine Mitte. Dass die Symmetrie bei $G_1, G_2, G_3$ und $G_4$ auftritt, lässt vermuten, dass sie für $G_n$, mit $n \ge 1$, allgemein gilt.
Bei genauerer Betrachtung des Effekts von $\gamma_i$ auf zwei zur Mitte symmetrisch liegende Permutationen fällt noch eine stärkere Eigenschaft auf. Wir betrachten als Beispiel die zwei symmetrisch liegenden Permutationen $p = 1,3,4,2$ und $p^* = 4, 2, 1, 3$:
\begin{align*}
    \gamma_1 p = 2, 3, 1 \quad \gamma_1 p^* = 2, 1, 3 \\
    \gamma_2 p = 1, 3, 2 \quad \gamma_2 p^* = 3, 1, 2 \\
    \gamma_3 p = 3, 1, 2 \quad \gamma_3 p^* = 1, 3, 2 \\
    \gamma_4 p = 3, 2, 1 \quad \gamma_4 p^* = 1, 2, 3
\end{align*}
Die Ergebnisse der jeweiligen Anwendung von $\gamma_i$ liegen symmetrisch um die Mitte der Liste aller Permutationen von Länge 3. Mithilfe des Programms \emph{test\_sym.cpp} konnte diese Eigenschaft von $\gamma_i$ für alle symmetrisch gelegenen Paare an Permutationen bis Größe 10 bestätigt werden. Es scheint also, dass die Symmetrie des Pancake-Graphen allgemein gültig ist.

Um diese Vermutung zu beweisen, sind einige weitere Mittel nötig. Zunächst wird das fakultätsbasierte Zahlensystem eingeführt, mithilfe dessen der Effekt des $\gamma_i$-Operators aus einer völlig anderen Perspektive betrachtet werden kann. Denn das fakultätsbasierte Zahlensystem erlaubt es, die Anzahl an Inversionen symmetrisch gelegener Paare von Permutationen miteinander in Verbindung zu bringen. Damit kann schließlich gezeigt werden, dass die Anwendung von $\gamma_i$ auf zwei in einer lexikographisch sortierten Liste symmetrisch positionierte Permutationen wieder zu einem symmetrischen Paar führt.

\subsubsection*{Das fakultätsbasierte Zahlensystem}

Im fakultätsbasierten Zahlensystem wird im Gegenstz zum Dezimal- oder Binärsystem eine unterschiedliche Basis für jede Ziffer verwendet. Die $k$-te Ziffer (mit 0 beginnend), von rechts gelesen, verwendet $k!$ als Basis und kann die Werte 0 bis $k$ annehmen. Der Wert einer fakultätsbasiert geschriebenen Zahl ist die Summe der einzelnen Ziffern, multipliziert mit ihrer jeweiligen Basis. Beispielsweise ist
\begin{align*}
    17_{10} & = 2210_!   = 2 \cdot 3! + 2 \cdot 2! + 1 \cdot 1! + 0 \cdot 0!             \\
    24_{10} & = 10000_! = 1 \cdot 4! + 0 \cdot 3! + 0 \cdot 2! + 0 \cdot 1! + 0 \cdot 0! \\
    23_{10} & = 3210_!  = 3 \cdot 3! + 2 \cdot 2! + 1 \cdot 1! + 0 \cdot 0!
\end{align*}
wobei das tiefgestellte ! auf das fakultätsbasierte Zahlensystem hinweist. Eine Fakultät $n!$, geschrieben im fakultätsbasierten Zahlensystem, ist immer von der Form $1000\dots$ ($n$ Nullen). Die Ziffern von $n! - 1$ sind immer genau $n-1, n-2, n-3, \dots, 0$.
Da sich mit einer fakultätsbasierten Zahl mit $n$ Ziffern genau $n!$ Zahlen darstellen lassen, können diese Zahlen auf natürlichem Weg zum Nummerieren von Permutationen der Länge $n$ verwendet werden. Die Folgenden zwei Arten der Nummerierung sind entscheidend den Beweis der Symmetrie des Pancake-Graphen. Beide bilden eine Bijektion zwischen Permutationen der Länge $n$ und ganzen Zahlen von 0 bis $n! - 1$.

\begin{definition}
    Mit $\mu(p)$ wird der Index der Permutation $p$ in einer \emph{lexikographisch aufsteigend} sortierten Folge aller Permutationen der Länge $|p|$ bezeichnet. Mit $\mu(p)_i$ wird die $i$-te Ziffer (beginnend von links) von $\mu(p)$, geschrieben im fakultätsbasierten Zahlensystem, bezeichnet.
\end{definition}

$\mu(p)$ ist auch als Lehmer-Code von $p$ bekannt \cite{factorial}.

\begin{definition}
    Mit $\nu(p)$ wird der Index der Permutation $p$ in einer \emph{kolexikographisch absteigend} sortierten Folge aller Permutationen der Länge $|p|$ bezeichnet. Mit $\nu(p)_i$ wird die $i$-te Ziffer (beginnend von links) von $\nu(p)$, geschrieben im fakultätsbasierten Zahlensystem, bezeichnet.
\end{definition}

Bei Sortierung nach kolexikographischer Ordung werden die Permutationen von rechts anstatt von links beginnend verglichen \cite{lexicographic}.

Für $\mu(p)$ und $\nu(p)$ gelten folgende Eigenschaften: $\mu(p)_i$ ist genau die Anzahl an kleineren Elementen rechts von $p_i$, und $\nu(p)_{|p| - i - 1}$ die Anzahl an größeren Elementen links von $p_i$ \cite{factorial}. Mit $\mu(p)$ ist es nun möglich, die zu $p$ symmetrisch gelegene Permutation zu definieren.

\begin{definition}
    Sei $p$ eine Permutation der Länge $n$. $p^*$ bezeichnet die Permutation, sodass $\mu(p) = n! - \mu(p^*) - 1$.
\end{definition}

$p$ und $p^*$ werden auch als symmetrisches Paar von Permutationen bezeichnet. Zum Beweis der Symmetrie von $G_n$ soll gezeigt werden, dass $\mu(\gamma_i p) + \mu(\gamma_i p^*) = (n - 1)! - 1$ ist. Denn daraus folgt direkt, dass $\gamma_i p$ und $\gamma_i p^*$ wieder ein symmetrisches Paar von Permutationen ist. Die Strategie dafür ist, zunächst zu zeigen, dass wenn ein Präfix zweier symmetrisch gelegener Permutationen umgekehrt wird, wieder zwei symmetrisch gelegene Permutationen entstehen. Dafür wird allerdings noch folgendes Lemma benötigt.

\begin{lemma}
    Sei $p$ eine Permutation der Länge $n$ und $p_i$ ein Element von $p$ $(0 \le i \le n - 1)$. Für jedes $0 \le j \le n-1, j \ne i$ ist entweder ($p_i < p_j$ und $p^*_i > p^*_j$) oder ($p_i > p_j$ und $p^*_i < p^*_j$).
\end{lemma}

\begin{proof}
    In anderen Worten sagt Lemma 1, dass die kleineren, rechts bzw. links gelegenen Elemente von $p_i$ und $p^*_i$ alle an unterschiedlichen Positionen liegen. Für jede Position $j$ ist also entweder $p_i$ und $p_j$ oder $p^*_i$ und $p^*_j$ eine Inversion. Zunächst soll eine etwas schwächere Eigenschaft gezeigt werden, die für den Beweis nötig ist.

    Aufgrund der Definition von $p^*$ gilt
    \begin{align*}
        \mu(p) + \mu(p^*) & = n! - 1                                                                           \\
                          & =(n - 1) \cdot (n - 1)! + (n - 2) \cdot (n - 2)! + \dots + 1 \cdot 1! + 0 \cdot 0!
    \end{align*}
    Die Koeffizienten der Fakultäten in der zweiten Zeile sind genau die Ziffern von $\mu(p) + \mu(p^*)$ in fakultätsbasierter Schreibweise, daher gilt
    \begin{align*}
        \mu(p)_i + \mu(p^*)_i = n - i - 1 \quad 0 \le i \le n - 1
    \end{align*}
    Ist das nicht der Fall, ist es leicht zu überprüfen, dass $\mu(p) + \mu(p^*) \ne n! - 1$. Daher muss die Anzahl an kleineren, rechts gelgenen Elementen von $p_i$ plus der Anzahl an kleineren, rechts gelegenen Elementen von $p^*_i$ genau $n-i-1$ sein. Mit ähnlicher Begründung kann gezeigt werden, dass $\nu(p)_i + \nu(p^*)_i = i$, folglich ist die Anzahl an größeren, links gelegenen Elementen von $p_i$ plus der Anzahl an größeren, links gelegenen Elementen von $p^*_i$ genau $i$. Die Anzahl an links gelegenen, größeren und rechts gelegenen, kleineren Elementen von $p_i$ und $p^*_i$ zusammengezählt ist also allein anbhängig von $i$ und unabhängig davon, welche Permutation $p$ ist.

    Nun wird der eigentliche Beweis durchgeführt, er funktioniert über unendlichen Abstieg. Man nehme an, dass für irgendein $j_0 > i$ sowohl $p_{j_0} < p_i$, als auch $p^*_{j_0} < p^*_i$ gilt. Der Fall $j_0 < i$ funktioniert ähnlich. Auch die Annahme $p_{j_0} < p_i$ und $p^*_{j_0} < p^*_i$ dient nur der einfacheren Beweisführung, der Fall $p_{j_0} > p_i$ und $p^*_{j_0} > p^*_i$ kann mit der gleichen Methode bewiesen werden. Da $\nu(p)_{n - j_0 - 1} + \nu(p^*)_{n - j_0 - 1} = j_0$, muss für irgendein ein $j_1 < j_0$ gelten, dass $p_{j_1} < p_{j_0}$ und $p^*_{j_1} < p^*_{j_0}$. Andernfalls wäre es nicht möglich, auf insgesamt $j_0$ links gelegene, größere Elemente von $p_{j_0}$ und $p^*_{j_0}$ zu kommen. Nun gibt es zwei Fälle:
    \begin{enumerate}
        \item $j_1 < i$: Rechts von $j_1$ liegen $i$ und $j_0$, das heißt, es muss zwei Indizes $j_2, j_3 > j_1$ geben, sodass $p_{j_2} < p_{j_1}$ und $p^*_{j_2} < p^*_{j_1}$, $p_{j_3} < p_{j_1}$ und $p^*_{j_3} < p^*_{j_1}$. Andernfalls wäre es wieder nicht möglich, auf die nötigen $\mu(p)_{j_1} + \mu(p^*)_{j_1} = n - j_1 - 1$ nötigen, kleineren, rechts gelegenen Elemente zu kommen. Denn bereits zwei der $n - j_1 - 1$ rechts gelegenen Plätze sind sowohl in $p$ als auch in $p^*$ durch größere Zahlen besetzt, aber $n - j_1 - 1$ kleinere Elemente sind erforderlich.
        \item $i < j_1 < j_0$: Da $p_{j_1} < p_{j_0}$ und $p^*_{j_1} < p^*_{j_0}$, muss es rechts von $j_1$ einen Index $j_2 > j_1$ geben, sodass $p_{j_2} < p_{j_1}$ und $p^*_{j_2} < p^*_{j_1}$. Auch links von $j_1$ muss es einen Index $j_3 < j_1$ geben, sodass $p_{j_3} < p_{j_1}$ und $p^*_{j_3} < p^*_{j_1}$. Erneut wäre andernfalls das Erreichen der nötigen $\mu(p)_{j_1} + \mu(p^*)_{j_1} = n - j_1 - 1$ rechts gelegenen, kleineren und der $\nu(p)_{n - j_1 - 1} + \nu(p^*)_{n - j_1 - 1} = j_1$ links gelegenen, größeren Elemente unmöglich.
    \end{enumerate}
    Man sieht, dass durch die Forderung nach einer allein vom Index anbhängigen Zahl rechts gelegener, kleinerer bzw. links gelegener, größerer Elemente immer kleinere Zahlen in $p$ und $p^*$ gezwungen werden. Um diese herum sind aber nur größere Elemente, wodurch wieder kleinere Zahlen zum Ausgleich nötig werden. Dieser Prozess endet niemals, da mit jedem Schritt immer noch kleinere Zahlen erzeugt werden. Das ist ein Widerspruch, da natürliche Zahlen, wie sie in einer Permutation vorkommen, nicht unendlich oft verringert werden können. (Daneben wären die $n$ verfügbaren Elemente irgendwann ausgeschöpft.) Das zeigt, dass die Annahme $p_{j_0} < p_i$ und $p^*_{j_0} < p^*_i$ falsch war.
\end{proof}

Nun wird die Funktion $\eta(p, i, j, k)$ eingeführt, die die Anzahl an Elementen von $p$ mit Indizes im Intervall $[i, j]$ zählt, die kleiner als $p_i$ sind.
\begin{definition}
    Sei $p$ eine Permutation der ersten $n$ natürlichen Zahlen. Wir definieren die Funktion
    $$
        \eta(p, i, j, k) = \sum_{h = j}^k
        \begin{cases}
            1 & \text{wenn } p_i > p_h   \\
            0 & \text{wenn } p_i \le p_h
        \end{cases}
    $$
    für $0 \le i, j, k \le n - 1$.
\end{definition}
Lemma 1 ermöglicht es, die Summe von $\eta$ für $p$ und $p^*$ einfach zu berechnen, denn für jeden Index ist entweder das Element in $p$ oder in $p^*$ kleiner als $p_i$ bzw. $p^*_i$.
\begin{align*}
    \eta(p, i, j, k) + \eta(p^*, i, j, k) & = \sum_{h = j}^k
    \begin{cases}
        1 & \text{wenn } p_i > p_h   \\
        0 & \text{wenn } p_i \le p_h
    \end{cases} + \sum_{h = j}^k
    \begin{cases}
        1 & \text{wenn } p^*_i > p^*_h   \\
        0 & \text{wenn } p^*_i \le p^*_h
    \end{cases}                                 \\
                                          & = \sum_{h = j}^k \Bigg (
    \begin{cases}
            1 & \text{wenn } p_i > p_h   \\
            0 & \text{wenn } p_i \le p_h
        \end{cases} +
    \begin{cases}
            1 & \text{wenn } p^*_i > p^*_h   \\
            0 & \text{wenn } p^*_i \le p^*_h
        \end{cases} \Bigg )                                 \\
                                          & = \sum_{h = j}^k 1       \\
                                          & = k - j + 1
\end{align*}
Damit kann nun gezeigt werden, dass das Umkehren von Präfixen zweier symmetrisch gelegener Permutationen erneut zu zwei symmetrisch gelegene Permutationen führt. Das ist bereits sehr nahe am gewünschten Ergebnis, es muss lediglich noch das erste Element aus beiden Permutationen entfernt werden.
\begin{lemma}Seien $x$ und $y$ die Permutationen, die aus $p$ und $p^*$ durch Umkehrung des Präfixes bis einschließlich Index $i$ hervorgehen. Wenn $\mu(p) + \mu(p^*) = n! - 1$, dann gilt auch $\mu(x) + \mu(y) = n! - 1$.
\end{lemma}

\begin{proof}
    Die Ziffern $\mu(p)$ und $\mu(p^*)$ geben in fakultätsbasierter Schreibweise die Anzahl an rechts gelegenen, kleineren Elementen an. Da an den Elementen $p_j, p^*_j$ für $j > i$ nichts geändert wird, ändert sich auch nicht ihre Zahl rechts gelegener, kleinerer Elemente, d. h. sie können im Folgenden außer Acht gelassen werden. Für $j < i$ werden alle links gelegenen, kleineren Elemente durch die Umkehrung auf die rechte Seite gebracht, die kleineren Elemente rechts von $i$ bleiben. Der Index, zu dem das $j$-te Element durch die Umkehrung bewegt wird, ist $i - j$. Folglich ist die Anzahl rechts gelegener, kleinerer Elemente von $x_{i - j}$ und $y_{i - j}$ zusammen
    \begin{align*}
        \mu(x)_{i - j} + \mu(y)_{i - j}
         & = \eta(p, j, 0, j - 1) + \eta(p, j, i + 1, n - 1) + \eta(p^*, j, 0, j - 1) + \eta(p^*, j, i + 1, n - 1) \\
         & = \eta(p, j, 0, j - 1) + \eta(p^*, j, 0, j - 1) + \eta(p, j, i + 1, n - 1) + \eta(p^*, j, i + 1, n - 1) \\
         & = j - 1 + 1 + n - 1 - (i + 1) + 1                                                                       \\
         & = j + n - i - 1                                                                                         \\
         & = n - (i - j) - 1
    \end{align*}
    In Zeile 1 wurden die vor der Umkehrung links gelegenen, kleineren Elemente zu den kleineren Elementen rechts von $i$ addiert. Daraus folgt
    \begin{align*}
        \mu(x)_i + \mu(y)_i = n - i - 1 \quad 0 \le i \le n - 1
    \end{align*}
    und damit
    \begin{align*}
        \mu(x) + \mu(y) = n! - 1
    \end{align*}
\end{proof}

Man sieht, dass das Umkehren eines Präfixes in einem symmetrischen Paar von Permutationen wieder zu einem symmetrischen Paar führt. Nun kann die Symmetrie des Pancake-Graphen, bzw. des $\gamma_i$-Operators als Satz festgehalten werden.

\begin{theorem}
    Wenn $p$ eine Permutation der Länge $n$ und $q$ eine Permutation der Länge $n - 1$ ist, gilt
    $$
        \gamma_i p = q \Longleftrightarrow \gamma_i p^* = q^* \quad 0 \le i \le n - 1
    $$
\end{theorem}

\begin{proof}
    Wir nennen die Permutation, die man durch Umkehren des Präfixes bis $i$ von $p$ erhält, $x$. Aufgrund von Lemma 2 ist $x^*$ genau die Permutation, die man durch Umkehren des Präfixes bis $i$ von $p^*$ erhält.
    Da
    \begin{align*}
        \gamma_i p = x_1, x_2, \dots, x_{n-1} \\
        \gamma_i p^* = x^*_1, x^*_2, \dots, x^*_{n-1}
    \end{align*}
    entspricht die Anzahl rechts gelegener, kleinerer Elemente von $(\gamma_i p)_0, (\gamma_i p)_1, \dots, (\gamma_i p)_{n - 2}$ bzw. $(\gamma_i p^*)_0, (\gamma_i p^*)_1, \dots, (\gamma_i p^*)_{n - 2}$ genau der Anzahl rechts gelegener, kleinerer Elemente von $x_1, x_2, \dots, x_{n-1}$ bzw. $x^*_1, x^*_2, \dots, x^*_{n-1}$. Daher gilt
    \begin{align*}
        \mu(\gamma_i p)_i = \mu(x)_{i + 1} \\
        \mu(\gamma_i p^*)_i = \mu(x^*)_{i + 1}
    \end{align*}
    und folglich
    \begin{align*}
        \mu(\gamma_i p)_i + \mu(\gamma_i p^*)_i & = \mu(x)_{i + 1} + \mu(x^*)_{i + 1} \\
                                                & = (n - (i + 1) - 1)                 \\
                                                & = (n - 1) - i - 1
    \end{align*}
    Da $|\gamma_i p| = |\gamma_i p^*| = n - 1$, ist $\gamma_i p$ und $\gamma_i p^*$ ein symmetrisches Paar von Permutationen der Länge $n - 1$. Somit gilt $\gamma_i p = q$ genau dann, wenn $\gamma_i p^* = q^*$.
\end{proof}

\subsection{Reduktion des Burnt Pancake-Problems}

In diesem Abschnitt soll gezeigt werden, dass $\gamma_i$-Pancake Sort mindestens so schwierig wie das Burnt Pancake Problem, eine Variante des Pfannkuchen-Sortierproblems, ist. Mit ``mindestens so schwierig'' ist gemeint, dass das Burnt Pancake Problem in polynomieller Zeit auf das vorliegende Problem reduziert werden kann. Damit wird gezeigt, dass wahrscheinlich kein Algorithmus existiert, der $\gamma_i$-Pancake Sort in polynomieller Zeit löst. Für das Burnt Pancake-Problem gibt es zwar keinen Beweis der NP-Schwere, allerdings ist es seit über 40 Jahren niemandem gelungen, einen Algorithmus mit polynomieller Laufzeit für dieses Problem zu entwickeln.

Im Burnt Pancake Problem haben Pfannkuchen eine verbrannte Seite, die nach dem Sortieren bei jedem Pfannkuchen unten liegen muss \cite{burntpancakes}. Indem das Konzept der verbrannten Seite durch ein Vorzeichen vor jedem Element der Permutation formalisiert wird, kann das Burnt Pancake-Problem wie folgt definiert werden.
\begin{definition}[Burnt Pancake-Problem]
    Gegeben sei eine vorzeichenbehaftete Permutation
    \begin{align*}
        p = \sigma_0 p_0, \sigma_1 p_1, \dots, \sigma_{n-1} p_{n-1}
    \end{align*}
    wobei $1 \le p_i \le n$ und $\sigma_i \in \{-1, 1\}$ für $0 \le i \le n - 1$ sowie $p_i \ne p_j$ für $i \ne j$. Was ist die minimale Anzahl an Präfixumkehrungen, wobei bei einer Präfixumkehrung auch alle Vorzeichen im umgekehrten Präfix invertiert werden, sodass $p$ in die identische Permutation der Länge $|p|$ mit ausschließlich positiven Elementen überführt wird?
\end{definition}
Wenn $\sigma_i = -1$ ist, bedeutet das, dass der $i$-te Pfannkuchen die verbrannte Seite oben hat.

\begin{theorem}
    $\gamma_i$-Pancake Sort ist mindestens so schwierig wie das Burnt Pancake-Problem.
\end{theorem}

\begin{proof}
    Es wird gezeigt, dass schon das Bestimmen der minimalen Anzahl an $\gamma_i$-Operationen mindestens so schwierig wie das Burnt Pancake-Problem ist. Daraus folgt direkt, dass auch $\gamma_i$-Pancake Sort mindestens so schwierig ist, denn mit jedem Algorithmus, der $\gamma_i$-Pancake Sort löst, kann man die minimale Anzahl an $\gamma_i$-Operationen bestimmen, indem man die Länge der zurückgegebenen Folge an $\gamma_i$-Operationen bestimmt.

    Die Reduktion geschieht nun, indem das Burnt Pancake-Problem in $\gamma_i$-Pancake Sort simuliert wird. Für eine vorzeichenbehaftete Permutation $p$ der Länge $n$, die Eingabe für das Burnt Pancake-Problem, konstruieren wir eine nicht-vorzeichenbehaftete Permutation $q$ wie folgt:
    $$
        q = a_0, a_1, \dots, a_{n-1}
    $$
    $$
        a_i  = \begin{cases}
            (p_i - 1) \cdot 3n + 1, (p_i - 1) \cdot 3n + 2, \dots, p_i \cdot 3n & \text{wenn } \sigma_i = 1  \\
            p_i \cdot 3n, p_i \cdot 3n - 1, \dots, (p_i - 1) \cdot 3n + 1       & \text{wenn } \sigma_i = -1
        \end{cases}
    $$
    Jedem Pfannkuchen in $p$ wird je nach seiner Orientierung eine aufsteigende oder absteigende Folge von $3n$ aufeinanderfolgenden, natürlichen Zahlen zugeordnet. Dieses Vorgehen ist in Abbildung 2 veranschaulicht. Mit $a_i$ wird die zu $p_i$ zugehörige Teilfolge von $q$ bezeichnet. Der Einfachheit halber bezeichnet $a_i$ auch die zu $p_i$ zugehörige Teilfolge in $q$ nach einigen Sortierschritten, auch wenn diese dann andere Zahlen enthalten kann bzw. gekürzt worden sein kann. Die Folgen $a_0, a_1, \dots, a_{n-1}$ in Zeile 1 werden zu einer Folge $q$ zusammengefügt, $q$ ist also keine Folge von Folgen. Die Länge von $q$ beträgt $n \cdot 3n = 3n^2$, ist also durch ein Polynom in $n$ beschränkt, d. h. die Reduktion ist eine Polynomialzeitreduktion.

    \begin{figure}[h]
        \begin{center}
            \includegraphics[width=\textwidth]{grafiken/burnt-pancake-reduktion}
            \caption{Reduktion des Burnt Pancake-Problems auf $\gamma_i$-Pancake Sort. Die Pfeile geben die Orientierung der verbrannten Pfannkuchen an, die Pfeilspitze zeigt zur nicht verbrannten Seite.}
        \end{center}
    \end{figure}
    Ist $\sigma_i = 1$, d. h. der $i$-te Pfannkuchen ist richtig orientiert, wird er durch eine aufsteigende Folge repräsentiert. Das entspricht auch der richtigen Orientierung im Sortieren ohne Vorzeichen. Denn wäre $p$ vollständig sortiert und $\sigma_i = 1$ für alle $0 \le i \le n - 1$, wäre auch $q$ vollständig sortiert. Die Idee ist nun, zu zeigen, dass kein $a_i$ in einer optimalen Folge an $\gamma_i$-Operationen in der Mitte getrennt oder vollständig aufgegessen wird. Denn dann werden durch $\gamma_i$-Operationen immer nur Pfannkuchen am Anfang und Ende einer der Folgen $a_i$ entfernt, sodass sich diese schließlich wie ein verbrannter Pfannkuchen verhalten. Durch die $3n$ Pufferelemente bleiben von jedem $a_i$ außerdem mindestens $n$ Elemente übrig, sodass das Ergebnis des Sortierens mit $\gamma_i$ als sortierter Stapel an verbrannten Pfannkuchen interpretiert werden kann.

    Die einem Element von $p$ zugeordnete Teilfolge von $q$ wird niemals durch eine optimale Folge an $\gamma_i$-Operationen vollständig entfernt, da eine vorzeichenbehaftete Permutation in maximal $2n$ Schritten sortiert werden kann. Diese obere Schranke stammt von Cohen und Blum \cite{burntpancakes}. Für $n \ge 10$ wurde dort sogar eine Schranke von $2n - 2$ angegeben, der Einfachheit halber wird im Folgenden allerdings mit $2n$ gearbeitet. Es folgt, dass eine Folge an $\gamma_i$-Operationen, die länger als $2n$ ist, nicht optimal sein kann, denn man könnte $q$ in maximal $2n$ Schritten sortieren, indem man eine optimale Folge an Präfixumkehrungen für $p$ simuliert. Wenn in $p$ das Präfix bis Index $i$ umgekehrt wird, wendet man $\gamma_j$ auf $q$ an, wobei $j$ die größtmöglich Zahl ist, sodass $q_j$ zur Teilfolge von $p_i$ gehört (unter Beachtung der Tatsache, dass durch jede Anwendung von $\gamma_i$ möglicherweise einige Elemente in $q$ verringert werden). Da jedes Element von $p$ nach der Sortierung positiv ist, sind nach der Definition von $q$ auch alle den Elementen von $p$ zugeordneten Teilfolgen aufsteigend sortiert. Da die Pfannkuchen in $p$ aufsteigend sortiert sind, sind auch die ihnen zugeordneten Teilfolgen in $q$ aufsteigend sortiert, daher lässt sich $q$ tatsächlich in maximal $2n$ Schritten sortieren. Da also maximal $2n$ $\gamma_i$-Operationen durchgeführt werden und mit jeder $\gamma_i$-Operation genau ein Element aus $q$ entfernt wird, muss jedes $a_i$ nach dem Sortieren noch mindestens $n$ Elemente enthalten.

    Nun wird gezeigt, dass es unter den optimalen $\gamma_i$-Folgen zum Sortieren von $q$ immer eine gibt, die im Sortierprozess keines der $a_i$ in der Mitte trennt. Das ist notwendig, da ein solcher Schritt, übertragen auf das Burnt Pancake-Problem, unmöglich wäre. Man nehme an, $a_j$ für $0 \le j \le n - 1$ wird durch $\gamma_k$ mit $l + 1 \le k \le r-1$ in zwei Teile geteilt, wobei $a_j$ vor der Anwendung von $\gamma_k$ in $q$ von $q_l$ bis $q_r$ reichte. Da nach vollständiger Sortierung von $q$ noch Elemente von $a_j$ vorhanden sein müssen, müssen die zwei Teile an einem bestimmten Punkt wieder in richtiger Orientierung zusammengefügt werden. Hätte man aber anstatt $\gamma_k$ $\gamma_r$ verwendet und sonst alle Operationen in gleicher Weise durchgeführt, wären die beiden Teile von $a_i$ ebenfalls in richtiger Orientierung zusammengefügt. Mit ,,in gleicher Weise'' ist gemeint, dass man hinter den gleichen $a_i$ wendet, wie man es mit Trennung von $a_j$ getan hätte, der genaue Index der $\gamma_i$-Operationen mag sich allerdings unterscheiden. Durch diese Veränderung wird die Gesamtzahl an $\gamma_i$-Operationen nicht vergrößert, daher gibt es immer eine optimale Folge an $\gamma_i$-Operationen ohne Trennung einer Folge $a_i$ in der Mitte.

    Es konnte gezeigt werden, dass jede Instanz des Burnt Pancake-Problems in polynomieller Zeit auf eine Instanz von $\gamma_i$-Pancake Sort reduziert werden kann. Damit ist $\gamma_i$-Pancake Sort mindestens so schwierig wie das Burnt Pancake-Problem.
\end{proof}

\subsection{Finden der kürzesten Folge an $\gamma_i$-Operationen}

\subsection{Berechnung der PWUE-Zahl}

\section{Implementierung}

\section{Quellcode}

\begin{thebibliography}{5}
    \bibitem{burntpancakes}
    Cohen, D. S., Blum, M. (1993).
    On the problem of sorting burnt pancakes. \\
    https://www.sciencedirect.com/science/article/pii/0166218X94000093

    \bibitem{discretemath}
    Johnsonbaugh, R. (2017).
    Discrete Mathematics (8. Auflage).
    Pearson Verlag.

    \bibitem{parallelbfs}
    Korf, R. E., Schultze, P. (2005).
    Large-Scale Parallel Breadth-First Search. \\
    https://www.aaai.org/Papers/AAAI/2005/AAAI05-219.pdf

    \bibitem{factorial}
    Wikipedia (2022).
    Factorial number system. \\
    https://en.wikipedia.org/wiki/Factorial\_number\_system

    \bibitem{lexicographic}
    Wikipedia (2022).
    Lexicographic order. \\
    https://en.wikipedia.org/wiki/Lexicographic\_order
\end{thebibliography}

\end{document}