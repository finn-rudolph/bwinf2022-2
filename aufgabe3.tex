\documentclass[a4paper, 11pt, ngerman]{article}
\usepackage{tikz-network}
\usepackage[left=2.5cm, right = 2.5cm,top=3cm,bottom=2.5cm, head=14pt]{geometry}
\usepackage{babel}
\usepackage[T1]{fontenc}
\usepackage[utf8x]{inputenc}
\usepackage{algpseudocode}


\newcommand{\Aufgabe}{Aufgabe 3: Pancake Sort} 
\newcommand{\TeilnahmeId}{67571}             
\newcommand{\Name}{Finn Rudolph}   

\usepackage{scrlayer-scrpage, lastpage}
\setkomafont{pageheadfoot}{\textrm}
\rohead{Teilnahme-ID: \TeilnahmeId}
\lohead{\Name}
\cfoot*{\thepage{}}

\usepackage{listings}
\usepackage{color}
\definecolor{mygreen}{rgb}{0,0.6,0}
\definecolor{mygray}{rgb}{0.5,0.5,0.5}
\definecolor{mymauve}{rgb}{0.58,0,0.82}
\definecolor{myblue}{rgb}{0.3, 0.0, 0.9}
\lstset{
language=C++,
  keywordstyle=\color{myblue},commentstyle=\color{mygreen},
  stringstyle=\color{mymauve},rulecolor=\color{black},
  basicstyle=\footnotesize\ttfamily,numberstyle=\tiny\color{mygray},
  captionpos=b, % sets the caption-position to bottom
  keepspaces=true, % keeps spaces in text
  numbers=left, numbersep=5pt, showspaces=false,showstringspaces=true,
  showtabs=false, stepnumber=2, tabsize=2, title=\lstname
}

\title{\Huge Aufgabe 3: Pancake Sort}
\author{\Large Finn Rudolph \\ \\ \Large Teilnahme-ID: 67571}
\date{\Large 23. Dezember 2022}

\begin{document}

\begin{titlepage}
    \maketitle
    \thispagestyle{empty}
\end{titlepage}

\tableofcontents
\thispagestyle{empty}
\newpage

\section{Lösungsidee}

\subsection{Der Pancake-Graph}

Wenn $p$ eine Permutation von $n$ Zahlen ist, bezeichnen wir mit $s(p)$ die Menge an Permutationen von $n - 1$ Zahlen, die aus $p$ durch eine Wende-und-Ess-Operation hervorgehen. Formal:
$$
    s(p) = \{q : q = \gamma_i p \} \quad 0 \le i < n
$$
Sei $\text{ind}(p)$ der Index von $p$ in einer lexikogrfisch sortierten Liste aller Permutationen der Größe $|p|$ (0-indexiert). Wir bezeichnen mit $p^*$ die Permutation, die in dieser Liste an Stelle $n! - \text{ind}(p) - 1$ steht, das heißt $p$ und $p^*$ stehen symmetrisch zur Mitte der Liste. Die Behauptung ist nun, dass auch die Permutationen $s(p)$ und $s(p^*)$ durch Spiegelung an der Mitte ineinander überführt werden können. Formal lässt sich das folgendermaßen erfassen,
$$
    q \in s(p) \Longleftrightarrow q^* \in s(p^*)
$$
wobei $q$ eine beliebige Permutation der Länge $|p| - 1$ ist.
\newline

\newpage
\begin{tikzpicture}[node distance = {19mm}, main/.style = {draw, circle}]
    \node[main](1234) at (0, 0) {1234};
    \node[main](1243) [below of = 1234] {1243};
    \node[main](1324) [below of = 1243] {1324};
    \node[main](1342) [below of = 1324] {1342};
    \node[main](1423) [below of = 1342] {1423};
    \node[main](1432) [below of = 1423] {1432};

    \node[main](2134) [below of = 1432] {2134};
    \node[main](2143) [below of = 2134] {2143};
    \node[main](2314) [below of = 2143] {2314};
    \node[main](2341) [below of = 2314] {2341};
    \node[main](2413) [below of = 2341] {2413};
    \node[main](2431) [below of = 2413] {2431};

    \node[main](123) at (7, -5.8) {123};
    \node[main](132) [below of = 123] {132};
    \node[main](213) [below of = 132] {213};
    \node[main](231) [below of = 213] {231};
    \node[main](312) [below of = 231] {312};
    \node[main](321) [below of = 312] {321};

    \node[main](12) at (11, -9.6) {12};
    \node[main](21) [below of = 12] {21};

    \node[main](1) at (13.5, -10.5) {1};

    \draw[->](1234) -- (123);
    \draw[->](1234) -- (213);
    \draw[->](1234) -- (321);

    \draw[->](1243) -- (132);
    \draw[->](1243) -- (213);
    \draw[->](1243) -- (321);

    \draw[->](1324) -- (213);
    \draw[->](1324) -- (123);
    \draw[->](1324) -- (231);

    \draw[->](1342) -- (231);
    \draw[->](1342) -- (132);
    \draw[->](1342) -- (312);
    \draw[->](1342) -- (321);

    \draw[->](1423) -- (312);
    \draw[->](1423) -- (123);
    \draw[->](1423) -- (231);

    \draw[->](1432) -- (321);
    \draw[->](1432) -- (132);
    \draw[->](1432) -- (312);
    \draw[->](1432) -- (231);

    \draw[->](2134) -- (123);
    \draw[->](2134) -- (312);

    \draw[->](2143) -- (132);
    \draw[->](2143) -- (123);
    \draw[->](2143) -- (312);

    \draw[->](2314) -- (213);
    \draw[->](2314) -- (132);

    \draw[->](2341) -- (231);
    \draw[->](2341) -- (321);

    \draw[->](2413) -- (312);
    \draw[->](2413) -- (213);
    \draw[->](2413) -- (132);

    \draw[->](2431) -- (321);
    \draw[->](2431) -- (231);

    \draw[->](123) -- (12);
    \draw[->](123) -- (21);
    \draw[->](132) -- (21);
    \draw[->](132) -- (12);
    \draw[->](213) -- (12);
    \draw[->](231) -- (21);
    \draw[->](312) -- (12);
    \draw[->](312) -- (21);
    \draw[->](321) -- (12);
    \draw[->](321) -- (21);

    \draw[->](12) -- (1);
    \draw[->](21) -- (1);
\end{tikzpicture}

\newpage
\begin{tikzpicture}[node distance = {19mm}, main/.style = {draw, circle}]
    \node[main](3124) at (0, 0) {3124};
    \node[main](3142) [below of = 3124] {3142};
    \node[main](3214) [below of = 3142] {3214};
    \node[main](3241) [below of = 3214] {3241};
    \node[main](3412) [below of = 3241] {3412};
    \node[main](3421) [below of = 3412] {3421};

    \node[main](4123) [below of = 3421] {4123};
    \node[main](4132) [below of = 4123] {4132};
    \node[main](4213) [below of = 4132] {4213};
    \node[main](4231) [below of = 4213] {4231};
    \node[main](4312) [below of = 4231] {4312};
    \node[main](4321) [below of = 4312] {4321};

    \node[main](123) at (7, -5.8) {123};
    \node[main](132) [below of = 123] {132};
    \node[main](213) [below of = 132] {213};
    \node[main](231) [below of = 213] {231};
    \node[main](312) [below of = 231] {312};
    \node[main](321) [below of = 312] {321};

    \node[main](12) at (11, -9.6) {12};
    \node[main](21) [below of = 12] {21};

    \node[main](1) at (13.5, -10.5) {1};

    \draw[->](3124) -- (123);
    \draw[->](3124) -- (213);

    \draw[->](3142) -- (132);
    \draw[->](3142) -- (231);
    \draw[->](3142) -- (312);

    \draw[->](3214) -- (213);
    \draw[->](3214) -- (123);

    \draw[->](3241) -- (231);
    \draw[->](3241) -- (312);

    \draw[->](3412) -- (312);
    \draw[->](3412) -- (321);
    \draw[->](3412) -- (132);

    \draw[->](3421) -- (321);
    \draw[->](3421) -- (132);

    \draw[->](4123) -- (123);
    \draw[->](4123) -- (312);
    \draw[->](4123) -- (132);
    \draw[->](4123) -- (213);

    \draw[->](4132) -- (132);
    \draw[->](4132) -- (321);
    \draw[->](4132) -- (213);

    \draw[->](4213) -- (213);
    \draw[->](4213) -- (312);
    \draw[->](4213) -- (132);
    \draw[->](4213) -- (123);

    \draw[->](4231) -- (231);
    \draw[->](4231) -- (321);
    \draw[->](4231) -- (213);

    \draw[->](4312) -- (312);
    \draw[->](4312) -- (231);
    \draw[->](4312) -- (123);

    \draw[->](4321) -- (321);
    \draw[->](4321) -- (231);
    \draw[->](4321) -- (123);

    \draw[->](123) -- (12);
    \draw[->](123) -- (21);
    \draw[->](132) -- (21);
    \draw[->](132) -- (12);
    \draw[->](213) -- (12);
    \draw[->](231) -- (21);
    \draw[->](312) -- (12);
    \draw[->](312) -- (21);
    \draw[->](321) -- (12);
    \draw[->](321) -- (21);

    \draw[->](12) -- (1);
    \draw[->](21) -- (1);
\end{tikzpicture}

\section{Implementierung}

\section{Quellcode}

\end{document}