\documentclass[a4paper, 10pt, ngerman]{article}
\usepackage{tikz-network}
\usepackage[left=3.5cm, right = 3.5cm, top=3.5cm, bottom=3.5cm, head=13.6pt]{geometry}
\usepackage{babel}
\usepackage[T1]{fontenc}
\usepackage{inputenc}
\usepackage[noend,nosemicolon,algoruled,noline]{algorithm2e}
\usepackage{amsmath}
\usepackage{amsthm}
\usepackage[nottoc,notlot,notlof]{tocbibind}
\usepackage{graphicx}
\usepackage{float}
\usepackage{enumitem}
\usepackage{listings}
\usepackage{color}
\usepackage{booktabs}

\newcommand{\Aufgabe}{Aufgabe 1: Weniger krumme Touren}
\newcommand{\TeilnahmeId}{67571}
\newcommand{\Name}{Finn Rudolph}

\usepackage{scrlayer-scrpage, lastpage}
\setkomafont{pageheadfoot}{\textrm}
\rohead{Teilnahme-ID: \TeilnahmeId}
\lohead{\Aufgabe}
\cfoot*{\thepage{}}

\title{\LARGE \textbf{\Aufgabe}}
\author{\large Finn Rudolph \\ \\ \large Teilnahme-ID: 67571}
\date{\large 11. Februar 2022}

\setlist[enumerate]{label*={\arabic*.}}

\definecolor{keyword}{rgb}{0.2, 0.0, 0.7}
\definecolor{comment}{rgb}{0.5, 0.5, 0.5}
\definecolor{number}{rgb}{0.5, 0.5, 0.5}
\definecolor{string}{rgb}{0.3, 0.65, 0.0}

\lstset
{
    basicstyle=\footnotesize\ttfamily,
    keywordstyle=\color{keyword},
    commentstyle=\color{comment},
    numberstyle=\tiny\color{number},
    stringstyle=\color{string},
    numbers=left,
    showspaces=false,
    showstringspaces=false
}

\begin{document}

\begin{titlepage}
    \maketitle
    \tableofcontents
    \thispagestyle{empty}
\end{titlepage}

\newtheorem{theorem}{Satz}
\newtheorem{lemma}{Lemma}
\theoremstyle{definition}
\newtheorem{definition}{Definition}

\section{Lösungsidee}

Das Problem wird durch einen Graphen modelliert. Jeder Punkt wird einem Knoten zugeordnet und zwischen jedem Knotenpaar existiert eine ungerichtete Kante, deren Gewicht die euklidsche Distanz zwischen den zugehörigen Punkten ist. Das Ziel ist es, einen möglichst kurzen Hamiltonpfad durch diesen Graphen zu finden. Die Bedingung, dass kein Abbiegewinkel von mehr als 90° vorkommen darf, bedeutet, dass bestimmte Tripel an Knoten nicht direkt aufeinander folgen dürfen. Das Problem soll nun als ganzzahliges lineares Programm formuliert werden und mithilfe eines IP-Solvers gelöst werden.

Im Folgenden wird mit $d_{ij}$ die Distanz zwischen Punkt $i$ und Punkt $j$, für $0 \le i, j \le n - 1$ bezeichnet, wobei $n$ die Anzahl an Punkten ist. Daneben bezeichnet $\vec{z_i}$ den zweidimensionalen Ortsvektor zu Punkt $i$. $\vec{z_{ij}}$ bezeichnet den Vektor $\vec{z_j} - \vec{z_i}$. Zuletzt ist $G$ der beschriebene Graph mit Knotenmenge $V$ und Kantenmenge $E$, sowie. Für die IP-Formulierung definieren wir die binären Variablen $x_{ij}$ für $0 \le i, j \le n - 1, i < j$, sodass $x_{ij} = 1$, wenn die Kante zwischen Punkt $i$ und Punkt $j$ in der optimalen Lösung verwendet wird, andernfalls $x_{ij} = 0$. Im Folgenden wird der Einfachheit halber manchmal $x_{ij}$ auch für $i > j$ geschrieben, gemeint ist immer $x_{ji}$. Das ganzzahlige lineare Programm sieht dann wie folgt aus.

\begin{align*}
    \text{minimiere} \quad & \sum_{i = 0}^{n - 1} \sum_{j = i + 1}^{n - 1} x_{ij} d_{ij} \\
    \text{sodass} \quad 
    & \sum_{j = 0, j \ne i}^{n - 1} x_{ij} \ge 1 & 0 \le i \le n - 1 & \quad (1) \\
    & \sum_{j = 0, j \ne i}^{n - 1} x_{ij} \le 2 & 0 \le i \le n - 1 & \quad (2) \\
    & \sum_{i = 0}^{n - 1} \sum_{j = i + 1}^{n - 1} x_{ij} = n - 1 & & \quad (3) \\
    & \sum_{i \in S} \sum_{j \in S, i < j} x_{ij} \le |S| - 1 & S \subseteq V, S \ne \emptyset & \quad (4) \\
    & x_{ij} + x_{jk} \le 1 & 0 \le i, j, k \le n - 1, i \ne j \ne k, i < k, \vec{z_{ij}} \cdot \vec{z_{jk}} < 0 & \quad (5) \\
    & x_{ij} \in \{0, 1\} & 0 \le i, j \le n - 1, i < j & \quad (6)
\end{align*}
\smallskip

Gleichungen (1) und (2) beschränkt den Grad jedes Knoten auf 1 oder 2. Gleichung (3) sorgt dafür, dass insgesamt $n - 1$ Kanten verwendet werden. Gleichung (4) ist der \emph{Subtour Elimination Constraint} in der Formulierung von Dantzig, Fulkerson und Johnson \cite{tsp-formulations}. Gleichung (5) setzt die Einschränkung des Abbiegewinkels um, wobei $\cdot$ das Skalarprodukt bezeichnet. Es wird nun gezeigt, wie die einzelnen Teile ihr Aufgabe jeweils erfüllen. Dafür sei $X = \{\{i, j\} : x_{ij} = 1\}$ die Menge an Kanten in der Lösung des ganzzahligen linearen Programms.

\begin{lemma}
    Der Graph $G' = (V, X)$ enthält keine Zyklen.
\end{lemma}

\begin{proof}
    Für einen Widerspruch nehme man an, dass sich ein Zyklus in $G'$ befindet, der genau die Knoten der Menge $T$ enthält. Dann gilt
    \begin{align*}
        \sum_{i \in T} \sum_{j \in T, i < j} x_{ij} \ge |T|
    \end{align*}
    da ein Zyklus aus $|T|$ Knoten genau $|T|$ Kanten enthält. Das ist ein Widerspruch zu (4) mit $S = T$, folglich war die Annahme, dass $G'$ einen Zyklus enthält, falsch.
\end{proof}

\begin{lemma}
    Die Kanten in $X$ bilden einen Hamiltonpfad in $G$.
\end{lemma}

\begin{proof}
    Wegen (3) enthält $G' = (V, X)$ genau $n - 1$ Kanten und wegen Lemma 1 ist $G'$ azyklisch. Ein azyklischer Graph mit $n - 1$ Kanten ist bekanntermaßen ein Baum mit $n$ Knoten, und damit ist $G'$ ein Spannbaum von $G$. Da der Grad jedes Knoten von (2) auf maximal 2 begrenzt wird, hat $G'$ genau zwei Blätter. Wählt man diese beiden Blätter als Endpunkte eines Pfads in $G$, erhält man einen Hamiltonpfad.
\end{proof}
 
Gleichung (1) ist zur Sicherstellung der gewünschten Eigenschaften von $X$ nicht nötig, verkürzte jedoch praktisch die Laufzeit, weshalb sie mit aufgeführt ist.

\begin{lemma}
    Jeder Abbiegewinkel zwischen zwei aneinanderliegenden Kanten in $X$ ist kleiner oder gleich $\pi / 2$.
\end{lemma}

\begin{proof}
    Der Abbiegewinkel $\alpha$ zweier Vektoren $\vec{u}$ und $\vec{v}$ ist der Betrag des Außenwinkels zwischen ihnen. Dieser ist mit dem Skalarprodukt durch folgende Identität verknüpft.
    \begin{align*}
        \cos(\alpha) = \frac {\vec{u} \cdot \vec{v}} {|\vec{u}||\vec{v}|}
    \end{align*}
    Da $|\alpha| \le \pi / 2 \Longleftrightarrow \cos(\alpha) \ge 0$ und $|\vec{u}| |\vec{v}| \ge 0$, ist $|\alpha| \le \pi / 2$ äquivalent zu $\vec{u} \cdot \vec{v} \ge 0$. Für einen Widerspruch nehme man an, dass der Betrag des Außenwinkels zwischen zwei unterschiedlichen Kanten $\{i, j\} \in X$ und $\{j, k\} \in X$ größer als $\pi / 2$ ist. Es wird außerdem angenommen, das $i < k$, was durch Tauschen von $i$ und $k$ immer erreicht werden kann. Dann gilt $x_{ij} + x_{jk} = 2$ und $\vec{z_{ij}} \cdot \vec{z_{jk}} < 0$, ein Widerspruch zu (5). 
\end{proof}

\section{Laufzeitanalyse}

\section{Implementierung}

\section{Beispiele}

\section{Quellcode}

\begin{thebibliography}{1}
    \bibitem{tsp-formulations}
    Öncan, T., Altinel, I. K., Laporte, G. (2009).
    A comparative analysis of several asymmetric traveling salesman problem formulations. \\
    https://mate.unipv.it/~gualandi/famo2conti/papers/tsp\_formulations.pdf
\end{thebibliography}

\end{document}